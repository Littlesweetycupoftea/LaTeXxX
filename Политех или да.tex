\documentclass[12pt]{article}
\usepackage[utf8]{inputenc}
\usepackage[russian]{babel}
\renewcommand{\baselinestretch}{1.4}
\pagestyle{empty}
\renewcommand{\rmfamily}{pnc}
\date{}
\begin{document}

    
    С самого 10 класса родители пытались выяснить, что же мне хотелось от этой жизни. Но я не сдавалась: я не говорила до последнего, так как сама не знала точно. Я хотела не очень многого: поступить на специальность, связанную с физикой, и учится с интересными и умными людьми . Было множество вариантов в Москве, и по одному в Петербурге и Екатеринбурге. Я очень надеялась на Москву, но обстоятельства решили пойти совсем иначе. Спонтанно решив, что Питер будет удобнее для посещения, стали искать нужный вуз. И, о боже! Здравствуй,  Политех! За 10 дней до начала обучения был выбран ИФНиТ и специальность "с меньшим количеством физики". Так я и попала в это чудесное учреждение.
     
    Честно, я ни разу не пожалела о том, что поступила именно в Политех. Довольно много нужного материала (от простого создания проекта, до интегрирования матриц), в принципе понятного и доступного. И я верю, что всё это пригодится мне в будущем для создания карьеры. 
    
    Карьера? Если честно, я не уверена, что буду работать по специальности. Но если такой шанс выпадет, я хочу начать с компании Газпром. Почему? Меня привлекает высокий уровень компании на рынке всего мира, очень много различных акций, творческих и спортивных состязаний (среднего уровня), концертов, мастер-классов и прочих интересных мероприятий для работников. А ещё, если ты очень хорош в своей профессии, то есть все шансы идти вверх по карьерной лестнице, получать премии и прочие профиты. С компанией определённость есть, но со специальностью обстоит сложнее. Я ещё не знаю, кем бы мне хотелось работать, что делать и сколько времени вкладывать в свою деятельность.
    
    Но уж если всё совсем не сложится ничего с нужным после вуза направлением, то я пойду в музыкальное училище. 
\end{document}